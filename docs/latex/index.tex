\subsection*{Requirements}


\begin{DoxyItemize}
\item Boost v1.\+53+ \mbox{[}check\+\_\+deleted, atomicadd, intrusive\+\_\+ptr\mbox{]}
\item C++11
\item lapack
\item blas
\item gcc v6+
\end{DoxyItemize}

\mbox{[}C\+U\+DA support\mbox{]}
\begin{DoxyItemize}
\item C\+U\+DA v10+
\item cu\+D\+NN
\end{DoxyItemize}

\mbox{[}Open\+Mp support\mbox{]}
\begin{DoxyItemize}
\item openmp
\end{DoxyItemize}

\mbox{[}Python\mbox{]}
\begin{DoxyItemize}
\item pybind11 2.\+2.\+4
\item numpy $>$= 1.\+15
\end{DoxyItemize}

\subsection*{ubuntu}

sudo apt-\/get install libboost-\/all-\/dev

\subsection*{compile}


\begin{DoxyItemize}
\item compile \$make -\/\+Bj4
\item turn on D\+E\+B\+UG mode\+: \$make -\/\+Bj4 D\+E\+B\+U\+G\+\_\+\+Enable=1
\item turn on Open\+Mp accelerate \$make -\/\+Bj4 O\+M\+P\+\_\+\+Enable=1
\item turn on G\+PU accelerate \$make -\/\+Bj4 G\+P\+U\+\_\+\+Enable=1
\item turn on G\+P\+U+\+Open\+Mp accelerate \$make -\/\+Bj4 G\+P\+U\+\_\+\+Enable=1 O\+M\+P\+\_\+\+Enable=1
\item compile python wrapper \$make pyobj -\/\+Bj4 $<$args$>$
\end{DoxyItemize}

$<$args$>$ can be O\+M\+P\+\_\+\+Enable, G\+P\+U\+\_\+\+Enable, D\+E\+B\+U\+G\+\_\+\+Enable, M\+K\+L\+\_\+\+Enable.

Note\+: if M\+K\+L\+\_\+\+Enable=1, will enforce using icpc as compiler.

\subsection*{Objects\+:}


\begin{DoxyItemize}
\item Storage \mbox{[}binded\mbox{]}
\item Tensor \mbox{[}binded\mbox{]}
\item Bond \mbox{[}binded\mbox{]}
\item Accessor \mbox{[}c++ only\mbox{]}
\item Symmetry \mbox{[}binded\mbox{]}
\item Uni\+Tensor \mbox{[}x\mbox{]}
\end{DoxyItemize}

\subsection*{Documentation\+:}

\subsection*{Feature\+:}

\subsubsection*{Python x C++}

Benefit from both side. One can do simple prototype on python side and easy transfer to C++ with small effort!


\begin{DoxyCode}
\textcolor{comment}{// c++ version:}
\textcolor{preprocessor}{#include "cytnx.hpp"}
\hyperlink{classcytnx_1_1Tensor}{cytnx::Tensor} A(\{3,4,5\},cytnx::Type.Double,cytnx::Device.cpu)
\end{DoxyCode}



\begin{DoxyCode}
\textcolor{comment}{# python version:}
\textcolor{keyword}{import} cytnx
A =  \hyperlink{classcytnx_1_1Tensor}{cytnx.Tensor}((3,4,5),dtype=cytnx.Type.Double,device=cytnx.Device.cpu)
\end{DoxyCode}


\subsubsection*{1. All the Storage and Tensor can now have mulitple type support.}

The avaliable types are \+:

\tabulinesep=1mm
\begin{longtabu} spread 0pt [c]{*{3}{|X[-1]}|}
\hline
\rowcolor{\tableheadbgcolor}\textbf{ cytnx type }&\textbf{ c++ type }&\textbf{ Type object  }\\\cline{1-3}
\endfirsthead
\hline
\endfoot
\hline
\rowcolor{\tableheadbgcolor}\textbf{ cytnx type }&\textbf{ c++ type }&\textbf{ Type object  }\\\cline{1-3}
\endhead
cytnx\+\_\+double &double &Type.\+Double \\\cline{1-3}
cytnx\+\_\+float &float &Type.\+Float \\\cline{1-3}
cytnx\+\_\+uint64 &uint64\+\_\+t &Type.\+Uint64 \\\cline{1-3}
cytnx\+\_\+uint32 &uint32\+\_\+t &Type.\+Uint32 \\\cline{1-3}
cytnx\+\_\+int64 &int64\+\_\+t &Type.\+Int64 \\\cline{1-3}
cytnx\+\_\+int32 &int32\+\_\+t &Type.\+Int32 \\\cline{1-3}
cytnx\+\_\+complex128 &std\+::complex$<$double$>$ &Type.\+Complex\+Double \\\cline{1-3}
cytnx\+\_\+complex64 &std\+::complex$<$float$>$ &Type.\+Complex\+Float \\\cline{1-3}
\end{longtabu}


\subsubsection*{2. Storage}


\begin{DoxyItemize}
\item Memory container with G\+P\+U/\+C\+PU support. maintain type conversions (type casting btwn Storages) and moving btwn devices.
\item Generic type object, the behavior is very similar to python.
\end{DoxyItemize}


\begin{DoxyCode}
Storage A(400,Type.Double);
\textcolor{keywordflow}{for}(\textcolor{keywordtype}{int} i=0;i<400;i++)
    A.at<\textcolor{keywordtype}{double}>(i) = i;

Storage B = A; \textcolor{comment}{// A and B share same memory, this is similar as python }

Storage C = A.to(Device.cuda+0); 
\end{DoxyCode}


\subsubsection*{3. Tensor}


\begin{DoxyItemize}
\item A tensor, A\+PI very similar to numpy and pytorch.
\item simple moving btwn C\+PU and G\+PU\+:
\end{DoxyItemize}


\begin{DoxyCode}
Tensor A(\{3,4\},Type.Double,Device.cpu); \textcolor{comment}{// create tensor on CPU (default)}
Tensor B(\{3,4\},Type.Double,Device.cuda+0); \textcolor{comment}{// create tensor on GPU with gpu-id=0}


Tensor C = B; \textcolor{comment}{// C and B share same memory.}

\textcolor{comment}{// move A to gpu}
Tensor D = A.to(Device.cuda+0);

\textcolor{comment}{// inplace move A to gpu}
A.to\_(Device.cuda+0);
\end{DoxyCode}

\begin{DoxyItemize}
\item Type conversion in between avaliable\+: 
\begin{DoxyCode}
Tensor A(\{3,4\},Type.Double);
Tensor B = A.astype(Type.Uint64); \textcolor{comment}{// cast double to uint64\_t}
\end{DoxyCode}
 \begin{DoxyVerb}    * vitual swap and permute. All the permute and swap will not change the underlying memory
    * Use Contiguous() when needed to actual moving the memory layout.
\end{DoxyVerb}
 
\begin{DoxyCode}
Tensor A(\{3,4,5,2\},Type.Double);
A.permute\_(\{0,3,1,2\}); \textcolor{comment}{// this will not change the memory, only the shape info is changed.}
cout << A.is\_contiguous() << endl; \textcolor{comment}{// this will be false!}

A.contiguous\_(); \textcolor{comment}{// call Configuous() to actually move the memory.}
cout << A.is\_contiguous() << endl; \textcolor{comment}{// this will be true!}
\end{DoxyCode}
 \begin{DoxyVerb}    * access single element using .at
\end{DoxyVerb}
 
\begin{DoxyCode}
Tensor A(\{3,4,5\},Type.Double);
\textcolor{keywordtype}{double} val = A.at<\textcolor{keywordtype}{double}>(\{0,2,2\});
\end{DoxyCode}
 \begin{DoxyVerb}    * access elements with python slices similarity:
\end{DoxyVerb}
 
\begin{DoxyCode}
\textcolor{keyword}{typedef} Accessor ac;
Tensor A(\{3,4,5\},Type.Double);
Tensor out = A.get(\{ac(0),ac::all(),ac::range(1,4)\}); 
\textcolor{comment}{// equivalent to python: out = A[0,:,1:4]}
\end{DoxyCode}

\end{DoxyItemize}

\subsection*{Avaliable linear-\/algebra function (Keep updating)\+:}

\begin{DoxyVerb}  func    |   inplace | CPU | GPU  | callby tn 
----------|-----------|-----|------|-----------
  Add     |   x       |  Y  |  Y   |    Y
  Sub     |   x       |  Y  |  Y   |    Y
  Mul     |   x       |  Y  |  Y   |    Y
  Div     |   x       |  Y  |  Y   |    Y
  +,+=[tn]|   x       |  Y  |  Y   |    Y (Add_)
  -,-=[tn]|   x       |  Y  |  Y   |    Y (Sub_)
  *,*=[tn]|   x       |  Y  |  Y   |    Y (Mul_)
  /,/=[tn]|   x       |  Y  |  Y   |    Y (Div_)
  Svd     |   x       |  Y  |  Y   |    Y
  Inv     |   Inv_    |  Y  |  Y   |    Y
  Conj    |   Conj_   |  Y  |  Y   |    Y 
  Exp     |   Exp_    |  Y  |  Y   |    Y
  Eigh    |   x       |  Y  |  Y   |    Y 
\end{DoxyVerb}


\subsection*{Generators}

\begin{DoxyVerb}Tensor: zeros(), ones(), arange()
\end{DoxyVerb}


\subsection*{Example}

\begin{DoxyVerb}See test.cpp for using C++ .
See test.py for using python  
\end{DoxyVerb}


\subsection*{Author}

\begin{DoxyVerb}Kai-Hsin Wu kaihsinwu@gmail.com \end{DoxyVerb}
 