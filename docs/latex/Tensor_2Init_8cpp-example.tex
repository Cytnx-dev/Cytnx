\hypertarget{Tensor_2Init_8cpp-example}{}\section{Tensor/\+Init.\+cpp}
initialize a Tensor$<$h2$>$Example\+: \paragraph*{c++ A\+PI}


\begin{DoxyCodeInclude}
\{ 
    \textcolor{comment}{//1. Create a Tensor with }
    \textcolor{comment}{//  shape (3,4,5),}
    \textcolor{comment}{//  dtype =Type.Double [default],}
    \textcolor{comment}{//  device=Device.cpu [default]}
    Tensor A(\{3,4,5\});
    cout << A << endl;

    \textcolor{comment}{//2. Create a Tensor with }
    \textcolor{comment}{//  shape (3,4,5),}
    \textcolor{comment}{//  dtype =Type.Uint64,}
    \textcolor{comment}{//  device=Device.cpu [default],}
    \textcolor{comment}{//  [Note] the dtype can be any one of the supported type.}
    Tensor B(\{3,4,5\},Type.Uint64);
    cout << B << endl;

    \textcolor{comment}{//3. Initialize a Tensor with }
    \textcolor{comment}{//  shape (3,4,5),}
    \textcolor{comment}{//  dtype =Type.Double,}
    \textcolor{comment}{//  device=Device.cuda+0, (on gpu with gpu-id=0)}
    \textcolor{comment}{//  [Note] the gpu device can be set with Device.cuda+<gpu-id>}
    Tensor C(\{3,4,5\},Type.Double,Device.cuda+0);
    cout << C << endl;

    \textcolor{comment}{//4. Create an empty Tensor, and init later}
    Tensor D;
    D.Init(\{3,4,5\},Type.Double,Device.cpu);

\}

\end{DoxyCodeInclude}
 